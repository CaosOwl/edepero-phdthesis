% Appendix A

% variables
\newcommand{\appdira}{appendices/plots/appendixA}

\chapter{Dark matter formulas and calculations}

\label{AppendixA}

\section{Calculation of $\ae$ cross-section using the Weizsacker-Williams approximation}
\label{appA:sec:cross-section-wz}

In this section, we derive the cross-section of the Dark Bremsstrahlung process used by NA64 to calculate the signal yield:

\begin{equation}
  \label{eq:dp-cs}
  \frac{d\sigma(e(p) + Z(P_i) \to e(p') + Z(P_f) + A'(k))}{dE_{\DM}d\cos{\theta_{\DM}}}
\end{equation}

Following the method in \cite{jdb}. We use the WW approximation, where $k=(E_{\DM},\vec{k})$ is the momentum of the emitted Dark Photon and $\theta_{\DM}$ is the angle relative to the incoming momentum of the electron $p$ in the lab frame. In the scenario of a fixed target experiment, we take the target nucleus initially at rest $P_i = (M,0)$ and the primary electron with the nominal beam energy $p = (E_0, \vec{p})$.

Using WW approximation, in the frame of the electron we see rapidly moving atom sources emitting a cloud of effective photons that the primary electron interact with to emit an $\DM$. This means we can reduce the cross section to the one of a real photon scattering, i.e. $e(p) + \gamma(q) \to e(p') + A(k)$ where the photon transports the difference of energy between initial and final state of the nucleus $q = P_f - P_i$. In this picture we redefine Eq.\ref{eq:dp-cs}:

\begin{equation}
  \label{eq:dp-cs-ww}
  \begin{aligned}
    \frac{d\sigma(e(p) + Z(P_i) \to e(p') + Z(P_f) + A'(k))}{dE_{\DM}d\cos{\theta_{\DM}}} = &\left(\frac{\alpha \chi}{\pi} \right) \left(\frac{E_0 x \beta_{\DM}}{(1 - x)} \right) \\
    &\times \left( \frac{d\sigma(e(p) + \gamma(q) \to e(p') + \DM(k))}{d(p \cdot k)} \right)_{t = t_{min}}
   \end{aligned}
 \end{equation}

 Where $x=E_{\DM}/E_0$ is the fraction of energy transfer to $\DM$ and $t=-q^2$. For a given four-momentum $k$ the virtuality $t$ has a minimum value $t_{min}$ when $\vec{q}$ is collinear with the vector $(\vec{k} - \vec{p})$. In this collinear geometry, we can solve the mass shell condition $(q + p - k)^2 = m^2_e$ and $P^2_f = (P_i - q)^2$:

 \begin{equation}
   \label{eq:t-min}
   t_{min} = -q^2_{min} \approx \left(\frac{\rm{U}}{2E_0 (1 - x)} \right)
 \end{equation}

 Where we define $\rm{U}$ using only leading effects:

 \begin{equation}
   \label{eq:dp-u}
   \rm{U}(x, \boldsymbol{\theta}_{\DM} ) = E_0^2 \boldsymbol{\theta}_{\DM} x + m^2_{\DM} \frac{1-x}{x} + m^2_e x
 \end{equation}

 Using this kinematics, we can express the following variables:

 \begin{equation}
   \label{eq:dp-md-var}
   \begin{aligned}
     &-\widetilde{u} \equiv m_e^2 - u_2 = 2p\cdot k - m^2_{\DM} = \rm{U}\\
     &\widetilde{s}  \equiv s_2 - m_e^2 = 2p' \cdot k + m^2_{\DM} = \frac{\rm{U}}{1-x} \\
     &t_2 = (p - p')^2 = - \frac{\rm{U}x}{1-x} + m^2_{\DM}
   \end{aligned}
 \end{equation}

 where $s_2$, $u_2$ and $t_2$ are the Mandelstam variables of the process. We compute the cross section as function of the variable defined above \cite{jdb}:

 \begin{equation}
   \label{eq:dp-cs-comp}
   \begin{aligned}
     \frac{d \sigma}{d(p \cdot k)} = 2 \frac{d\sigma}{dt_2} &\approx \frac{1}{8 \pi (s_2 - m^2_e)} |M|^2 \\
     &=\frac{4 \pi \alpha^2 \epsilon^2}{\widetilde{s}^2} \left( \frac{\widetilde{s}}{-\widetilde{u}} + \frac{-\widetilde{u}}{\widetilde{s}} + \frac{s m^2_{\DM} t_2}{-\widetilde{u}\widetilde{s}} \right) \\
           &=\underline{(4 \pi \alpha^2 \epsilon^2) \frac{(1-x)}{\rm{U}^2} \left[ 1 + (1 - x)^2 + \frac{2(1-x)^2m^2_{\DM}}{\rm{U}^2} \left( m^2_{\DM} - \frac{\rm{U}x}{1-x} \right) \right]}
   \end{aligned}
 \end{equation}

 In the last step, the equation was expanded in the order $m^2_e$.

 Now that we calculated the cross section of the $2 \to 2$ process we can plug it in Eq.\ref{eq:dp-cs-ww} to reach a complete formula of the WW cross-section:

 \begin{equation}
   \label{eq:dp-cs-ww-num}
   \frac{1}{E^2_0 x} \frac{d\sigma_{WW}}{dx d\cos{\theta_{\DM}}} = (8 \alpha^3 \epsilon^2 \chi \beta_{\DM}) \left[ \frac{1 - x + x^2/2}{\rm{U}^2} + \frac{(1-x)^2m^2_{\DM}}{\rm{U}^4} \left(m^2_{\DM} - \frac{\rm{U}x}{1 - x} \right) \right]
 \end{equation}

 To obtain a differential cross-section for $dx$ we need to integrate the equation above for $\cos{\theta_{\DM}}$. There are two problems to this integral:

 \begin{itemize}
 \item In the limit $m^2_{\DM} \to 0$ the denominator $\rm{U}(x, \theta_{\DM} = 0)$ becomes singular in $x \to 0$, which is the standard Bremsstrahlung singularity of photons.   
 \item The term $\chi$ can depend on $\theta_{\DM}$.
 \end{itemize}

 The singularity is in most of the cases of interest regulated by the Dark Photon mass. For $m_{\DM} \gg m_e$ the denominator takes the form $\rm{U} \approx m^2_{\DM} \frac{1-x}{x}$.
 $\chi$ on the other hand is an effective flux of photons that is integrated from the minimum to maximum momentum exchange $t_{min}$/$t_{max}$. For a general electric form factor $G_2(t)$:

 \begin{equation}
   \label{eq:g-ff}
   \chi = \int^{t_{max}}_{t_{min}} dt \frac{t - t_{min}}{t^2} G_2(t)
 \end{equation}

 The form factor $G_1(t)$ is relevant only for a negligible amount of the case of interest and is here neglected \cite{Kim:1973he}. As discussed in \cite{Kim:1973he,RevModPhys.46.815}, the maximum momentum transfer can be set to $t_{max} \sim m^2_{\DM}$, where the matrix element of $2 \to 3$ drops dramatically. We divide the form factor in elastic and inelastic component \cite{jdb}:

 \begin{align}
   \label{eq:g-ff-el}
   &G_{2,el}(t) = \left( \frac{a^2 t}{1 + a^2t} \right)^2 \left( \frac{1}{1 + t/d}\right) Z^2\\
   \label{eq:g-ff-in}
   &G_{2,in}(t) = \left( \frac{a'^2 t }{1 + a'^2 t} \right)^2 \left( \frac{1 + t (\mu_p^2 -1) / (4m^2_p)}{(1 + t / (\SI{0.71}{\giga\electronvolt\squared})^4} \right) Z
 \end{align}

 The parametrization becomes uncertain for high masses of the Dark Photon, but they are reliable for the NA64 range of interest, $m^2_{\DM} < \SI{1}{\giga\electronvolt}$. Numerically, the factor $\chi / Z^2$ depends on the beam energy used, but can be set to be $\chi/Z^2 = \mathcal{L}og \approx 5-10$ for the NA64 case. As we can see, we can neglect in good approximation the angular dependence of $\chi$ using these formulas. Coming back to Eq.\ref{eq:dp-cs-ww-num}, we drop $m^2_e$ and perform the integration on the angle and obtain a form for the $dx$ differential cross-section. The result is Eq.\ref{eq:dm-diff-cross} that we reported in chapter \ref{chapter1}, with the flux of photon $\chi$ expressed as the constant $\mathcal{L}og$:

 \begin{equation}
   \label{eq:dp-cs-dx}
\frac{d\sigma_{WW}}{dx} \approx \frac{8 Z^2 \alpha^3 \epsilon^2 x}{m^2_{\DM}} \left( 1 + \frac{x^2}{3(1-x)} \right) \mathcal{L}og    
\end{equation}


\subsection{Production rate}
\label{appA:sec:production-rate}

To compute the production rate, we integrate the differential cross section obtained above in the scenario of a high-energy electron impacting a fixed-target of $T$ radiation lengths (see chapter \ref{chapter1}):


\begin{equation}
  \label{eq:dm-gy}
  \frac{dN}{dx} = N_e \frac{N_0 X_0}{A} \int_{E_{\DM}}^{E_0} \frac{dE_1}{E_1} \int_0^T dt I(E_1;E_0;t) \times E_0 \frac{d\sigma}{dx'}\Big|_{x' = E_{\DM}/E_1}
\end{equation}

We need first to parametrize the energy distribution of the electrons after $t$ radiation lengths. We use the parametrization \cite{jdb}:

\begin{equation}
  \label{eq:i-dist}
  I(E_1;E_0,t) \approx  \frac{1}{E_0} y^{bt-1} bt \quad T \gtrsim 1
\end{equation}

where we take $y = \frac{E_0 - E_1}{E_0}$ and $b=4/3$. With this we find:

\begin{equation}
  \label{eq:i-int}
  \widetilde{I} = \int^T_0 dt I(E_1; E_0, t) \approx \frac{1 + y^{bT}(bT\ln{y} - 1)}{E_0by(\ln{y})^2} \underbrace{\to}_{T \to \infty} \frac{1}{E_0by(\ln{y})^2}
\end{equation}

Where the last approximation is correct within 1\% for $T>7$ and $0.5 < y$. This equation accounts for the constant $C'$ that multiples the signal rate in Eq.\ref{eq:dm-rate}. We stress that although the scaling is well reproduced using this method, the numerical result has a significant uncertainty that depends on the exact scenario considered. For a precise computation of the signal yield, the exact tree-level cross-section is used instead, implemented inside a MC simulation to reproduce the em-shower and the efficiency of the cuts.



\section{The Tree-level corrections}
\label{appA:sec:cross-section-tl}

In the NA64 experiment, the cross section is computed using precise tree-level calculation instead of one based on the WW approximation \cite{DMsimulation}. We use the momenta notation of Eq.\ref{eq:dp-cs}, and we perform the calculations in the frame of reference where the vector $\bf{V} = \bf{p} - \bf{k}$ is parallel to the z-axis and the vector $\bf{k}$ is in the xz-plane. In this frame the polar and axial angles of $\bf{q}$ are $\theta_q$ and $\phi_q$ respectively. We can express the relevant cross-section using the formula:

\begin{equation}
  \label{eq:cs-tl}
  \frac{d\sigma}{dx d\cos{\theta_{\DM}}} = \frac{\epsilon^2 \alpha^3 |\bf{k}| E_0}{|\bf{p}| |\bf{k} - \bf{p}|} \cdot \int_{t_{min}}^{t_{max}} \frac{dt}{t^2} G^{el}_{2}(t) \cdot \int_0^{2\pi} \frac{d\phi_q}{2\pi} \frac{|A^{2 \to 3}_{\DM}|^22}{8 M^2}
\end{equation}

 where $|A^{2 \to 3}_{\DM}|$ is the matrix element of the Dark Bremsstrahlung. We define the boundary of the first integral as in the previous section, and we express the amplitude squared of the cross section as:

\begin{equation}
  \label{eq:cs-dp-amp}
  \begin{aligned}
    |A^{2 \to 3}_{\DM}|^2 = \frac{2}{\widetilde{s}^2\widetilde{t}^2} \Big( &2m^2_e (\mathcal{P}^2 t (\widetilde{s} + \widetilde{u})^2 - 4((\mathcal{P} \cdot p)\widetilde{u} + (\mathcal{P} \cdot p')\widetilde{s})^2  \\
      &+ m^2_{\DM}(\mathcal{P}^2 t (\widetilde{s} - \widetilde{u})^2 -4((\mathcal{P}\cdot p)\widetilde{u} + (\mathcal{P}\cdot p')\widetilde{s})^2 \\
      &+ \widetilde{s}\widetilde{u}(\mathcal{P}^2 ((\widetilde{s} + t)^2 + (\widetilde{u}+t)^2) - 4 t ((\mathcal{P} \cdot p)^2 + (\mathcal{P} \cdot p')^2) \Big)
  \end{aligned}    
\end{equation}

Where we used the Mandelstam variables defined in Eq.\ref{eq:dp-md-var} (without the last equivalence to $\rm{U}$) and:

\begin{equation}
  \label{eq:dp-md-var-2}
  \mathcal{P}^2 = 4M^2 + t, \quad \mathcal{P} \cdot p = 2 M E_0 k_0 - (\widetilde{s} + t)/2, \quad \mathcal{P} \cdot p' = 2 M (E_0 - k_0) + (\widetilde{u} - t)/2
\end{equation}

Next we perform the integration over $\phi_q$ to find \cite{DMsimulation}:

\begin{equation}
  \label{eq:dp-cs-tl-int}
  \int_0^{2\pi} \frac{d\phi_q}{2\pi} |A^{2 \to 3}_{\DM}|^2 = A^{(0)}_{\DM} + Y \cdot A^{(1)}_{\DM} + \frac{1}{W^{1/2}} \cdot A^{(-1)}_{\DM} + \frac{Y}{W^{2/3}} \cdot A^{(-2)}_{\DM}
\end{equation}

Where we divided the calculation of the cross-section as the sum of different amplitudes:

\begin{equation}
  \label{eq:dp-cs-tl-par}
  \begin{aligned}
    &A^{(-2)}_{\DM} = -8M (4E_0M - t(2E_0 + M ))(m^2_{\DM} + 2m^2_e)\\
    &A^{(-1)}_{\DM} = \frac{8}{\widetilde{u}} \Big[ M^2 (2t\widetilde{u} + \widetilde{u}^2 + 4E_0^2 (2(x-1)(m^2_{\DM} + 2m_e^2) - t((x-2)x + 2)\\
    &+ 2t(-m^2_{\DM} - 2m_e^2 + t)) - 2E_0 M t ((1-x)\widetilde{u} + (x-2)(m^2_{\DM} + 2m^2_e + t) + t^2 (\widetilde{u} - m^2_{\DM}) \Big] \\
    &A^{(0)}_{\DM} = \frac{8}{\widetilde{u}^2} \Big[ M^2 (2t\widetilde{u} + (t - 4E_0^2(x-1)^2)(m^2_{\DM} + 2m^2_e)) + 2E_0M t (\widetilde{u} - (x-1)(m^2_{\DM} + 2m^2_e)) \Big] \\
    &A^{(1)}_{\DM} = \frac{8M^2}{\widetilde{u}} \\
    &Y = q^2 + 2q_0 E_0 - 2 \frac{|\bf{q}| |\bf{p}|}{|\bf{p} - \bf{k}|} (|\bf{p} - \bf{k} \cos{\boldsymbol{\theta_{\DM}}})\cos{\boldsymbol{\theta_{\DM}}_q}, \quad W = Y^2 - 4 \frac{|\bf{q}|^2|\bf{p}|^2|\bf{k}|^2}{|\bf{q} - \bf{k}|^2} \sin{\boldsymbol{\theta_{\DM}}}^2\sin{\boldsymbol{\theta}_q}^2
  \end{aligned}
\end{equation}

To derive the total cross-section, Eq.\ref{eq:cs-tl} is computed numerically within the allowed range of $x$ and $\theta$ (which is shown as an explicit $t$ integration in the equation). For a benchmark calculation the invisible mode scenario was used, using Lead as target ($Z = 82$) and $E_0 = 100 \gev$. Fig.\ref{fig:dp-cs-tl} shows the results of these calculation as the differential cross-section calculated for different mass as function of $x$. For a cross-check, the classical Bremsstrahlung of gamma as implemented in Geant4 was compared to the ETL we calculated. It agreed remarkably with the above calculation when a mass $m_{\DM} = 0$ is used, for a total relative disagreement of 8\%. In NA64, the expression used avoids infrared divergencies as the Dark Photon is expected to be massive. Moreover, the large cut used in NA64 searches $E^{ECAL}_{miss} > 50$ $\gev$ cuts the part of the spectrum containing the divergency even for $\DM$ mass close to zero.

\begin{figure}[bth!]
  \centering
  \includegraphics[width=\textwidth]{\appdira/tree-level-na64-calc.pdf}
  \caption[Tree level differential cross-section for different masses]{Differential cross-section of $\DM$ production via Bremsstrahlung as function of $x=E_{\DM}/E_0$ for different masses and $\epsilon = 1$ calculated using tree-level calculations. Red line shows the same spectrum used in Geant4 for the standard Bremsstrahlung $e + Z \to e + Z + \gamma$. The target used was lead with a primary energy $E_0 = 100 \gev$ \cite{DMsimulation}}
  \label{fig:dp-cs-tl}
\end{figure}

\subsection{Implementation inside the Geant4 simulation}

To implement the Exact Tree Level (ETL), we define the following ratio (called $K$-factor):

\begin{equation}
  \label{eq:k-factor}
  K(m_{\DM},E_0,Z,A) = \sigma^{\DM}_{WW} / \sigma^{\DM}_{ETL}
\end{equation}

which is numerically the correction to the cross-section calculated in the WW approximation. This factor is introduced to avoid calculating the ETL cross-section at each step, which would slow down the simulation significantly. Instead, we correct the WW cross-section with the $K$-factor for some benchmark scenario beforehand, and then correct the WW using those with a Bi-linear interpolation. The $K$-factor is sampled for 6 different masses and 10 different energies $E_0$ of the primary. We show the numerical values of these benchmark points in Fig.\ref{fig:k-factor}.

\begin{figure}[bth!]
  \centering
  \includegraphics[width=\textwidth]{\appdira/k-factor.pdf}
  \caption{Ratio between cross-section calculated in the IWW approximation and using ETL computation $K = \sigma^{\DM}_{IWW} / \sigma^{\DM}_{ETL}$ described in Sec.\ref{appA:sec:cross-section-tl} for different Dark photon masses \cite{DMsimulation}}
  \label{fig:k-factor}
\end{figure}

%%% Local Variables:
%%% mode: latex
%%% TeX-master: "../PhDthesis"
%%% End: