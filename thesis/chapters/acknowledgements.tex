In the beginning, I would like to thank my supervisor Prof. Paolo Crivelli for the valuable guidance received. In good times and in bad times, he helped me with advice, knowledge, and criticism when needed. This thesis gained tremendously from his support, and I will never stop being grateful for that. 

I wish to say thank you to Prof. Andre Rubbia for giving me the opportunity to participate in this amazing project. Thanks to him I learned to never ignore the details and always aim for a deeper understanding.

Next, I would like to thank Prof. Sergei Gninenko. I always admired his enthusiasm for physics that never failed to emerge in any situation. His tremendous amount of knowledge was invaluable times and again. The challenge of a two months beam time was much more manageable thanks to his never-ending optimism.

Next, I wish to thank Prof. Gunther Dissertori for accepting being my co-examiner for my thesis. I believe his lecture on particle physics during my bachelor was one milestone that influenced my later study.

During this thesis, I had the pleasure of working with truly wonderful peoples. Laura Molina Bueno and Dipanwita Banerjee are special among them for following me the closest after my supervisor. Laura in particular has been invaluable these last months, I will never forget the help and encouragement she gave me when I was close to the end. All the bachelor and master students I supervised during my PhD deserve a mention as well, it was wonderful to work with them and part of the thesis would not be possible without their help. I hope I managed to give back to them something in return.

I would like to thank the full NA64 collaboration and in particular Mikhail Kirsanov, Vladimir Poliakov, Anton Karneyeu, and most of all Balint Radics for guiding me through all the details and technicalities encountered during this thesis. I cannot imagine NA64 without them, and I wish them the best of luck for all the work left to be done. A thanks goes also to Michael Hoesgen, Renat Dusaev, Bogdan Vasilishin, and Azzrali Vitali. Their company during beam time was a real pleasure and helped sweeten the time spent in front of the monitor. Michael Hoesgen and Renat Dusaev in particular helped me even more than that, thanks to the guidance they provided for GEM trackers and the NA64 software.

I wish to thank all the CERN staff, which made my work easier. In particular, I would like to mention Alberto Ribon and Vladimir Uzhinskiy. Without their help with the hadron models in Geant4 my work would have been nearly impossible.

I will never forget all the amazing peoples I met at ETH. Thank you for all the fun I had during this work, for laughing with me about my mistakes, and to let me laugh about yours in return. I discovered a lot about physics talking with you, and you always reminded me that there is always something new to be learned in this subject. I hope we will keep in touch, your friendship will always be special.

In the end, I am grateful to my parents, my relatives, my brother, my dear friends back in Italy and here in Zurich, and the special persons I met. There are no words to express how much I love all of them, and how much my character is a consequence of the impact they had on me. I will always be grateful to my mom and dad to support me in this adventure, first and foremost emotionally but also financially to avoid any distractions outside my studies. The joy of a nephew will always be an amazing memory that I will connect to this period as well, I dedicate this thesis to Stella, a ray of sun in these difficult times. She is yet another reason to work hard for the future.