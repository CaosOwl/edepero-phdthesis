\chapter*{Preface}
\addcontentsline{toc}{chapter}{Preface}
\label{preface}

Physics is a wonderful thing, I was always mesmerized by its ability to describe what surrounds us in a powerful and compact way, ultimately using this knowledge for the betterment of mankind. When I was young, equations were like an ancient magic language to me. I was not able to understand it yet, but I was absolutely impressed by knowing that people that could, were able to control a power that for a kid looked like beyond any form of comprehension, from making incredibly heavy objects flying faster than any bird or destroying an entire city by splitting something invisible for the naked eye. I always dreamed to become part of this exclusive club of magicians, and now that I am, I have to say: I got it all wrong! Physics requires a lot of thinking, a lot of reading, a lot of trial and error, in one word? A lot of work! What follows is my attempt to make a lasting contribution to this amazing field. It won't go to history, but I hope it will be important for some students that follow my step, and for sure it will be important to me.

So what is this thesis about? Dark Matter, if I had to use one word. One of the most prominent puzzles in all physics. With all our powerful and extremely precise models, we still have to explain more than 96\% of the matter in our universe. That is embarrassing! How could we miss all that? Turns out is not simple at all when the matter that you are searching for stubbornly refuses to interact with your detectors, but I am sure that eventually, physics will prove to be even more stubborn in their measurement. This thesis was an attempt to this, an experiment to produce this elusive matter directly, and then measure its properties. Since my group does not have a noble price in its hand, it must be no mystery that we did not yet succeed, although this is not excluded for the future. I think the journey of me and the rest of my colleagues is nevertheless very instructive, it is the typical story of how an experiment is born and conducted. A "happy ending" is not always expected, and should not be assumed by the scientist that is conducting the experiment. He is there to observe, not decide. In my opinion, the experiment itself is the "happy ending" that a scientist is seeking, when it is well performed, and that is not depending on the outcome of it. So with no further wait, we can begin our journey, at the discovery of the NA64 experiment and its search of Dark Matter using the SPS\footnote{Super Proton Synchrotron}.

In this first chapter, our story begins, and as every story of science, it begins with a mystery: Dark Matter. Why can't we see it? Why can't we touch it? Those are not interesting questions, since many well-understood phenomena can't be seen or touched. In the end, it all boils down to one question: why our models, that work so amazingly well in so many different situations, fail miserably in other situations apparently equivalent? This is what this chapter will be about, understand why we need the concept of dark matter, and justify what phenomena it could explain. Indeed one could think that it is rather lazy to explain phenomena just by adding invisible matter to the system. "Just admit that you don't know what is going on!", is something that I heard myself when I try to explain my work to others. That is overall an honest question, a healthy scientific skepticism about a theory that seems so arbitrary, but that I hope it will be clarified after reading this thesis.

Chapter \ref{chapter1} is meant to just answer the question: what are we searching for? This is the beginning of every scientific experiment. Building an experiment to produce the Dark Photon is the next step, which will be covered in chapter \ref{chapter2}. Turn a photon in a Dark Photon using this portal is however meaningless if we can't prove that we did it! A robust analysis method of the data collected needs to be performed for this purpose, this is what we will explore in chapter \ref{chapter3} step by step, from the method to the selection criteria. In chapter $\ref{chapter4}$ we will then provide what most physicists are here for, the results! No Dark Matter has been found yet, but the data acquired allows us to exclude some specific models from being a credible explanation of reality. We now know thanks to this data, that the anomalous magnetic moment of the muon cannot be explained exclusively by the $\textrm{U'(1)}$. The NA64 experiment is however far from over! The stop caused by the LHC long shutdown (and unfortunately by the recent rise of Covid-19 as well) of the accelerator allowed us to look for ways to improve our setup for the upcoming challenges of 2021! The pieces of knowledge we gained for the background and the experimental condition are used to design the new version of the setup to acquire a larger number of particles and thus probe a larger number of models. In chapter \ref{chapter5} we will review all these changes, in particular the new setup for the visible mode of 2021, that was the last project of my Ph.D., which we will use to hopefully find (or else exclude) the $\DMX$.

With no further wait, let's embark on this journey!

%%% Local Variables:
%%% mode: latex
%%% TeX-master: "../PhDthesis"
%%% End: