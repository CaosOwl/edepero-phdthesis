% Chapter 1

\chapter{Data analysis} % Main chapter title
\label{chapter3} % For referencing the chapter elsewhere, use \ref{Chapter1} 

% ----------------------------------------------------------------------------------------

\section{General Analysis approach}
\label{chapter3:sec:analysis-approach}

\section{Geant4 simulation of the experiment}
\label{chapter3:sec:geant4}

\subsection{Correction of longitudinal shower profile in the Hadronic calorimeter}
\label{chapter3:sec:geant4-hcal-corr}

\subsection{Output digitization}
\label{chapter3:sec:geant4-digitization}

\subsection{Signal simulation}
\label{chapter3:sec:geant4-signal}

\section{Background}
\label{chapter3:sec:bkg}

\subsection{Heavy charged particle rejection using synchrotron radiation}
\label{chapter3:sec:bkg-srd}

\subsection{Hadron rejection using electromagnetic shower profile}
\label{chapter3:sec:bkg-ecal-profile}

\subsection{Study of K$^0_S$ background in visible mode}
\label{chapter3:sec:bkg-k0s}

\subsection{Study of neutral punch-through in the Calorimeters}
\label{chapter3:sec:bkg-neutrals}

\section{Study of $\gamma + Z \rightarrow \mu^+ \mu^-$ events }
\label{chapter3:sec:dimuons}


\subsection{Vertex and angle reconstruction}
\label{chapter3:sec:dimuons-reco}

\subsection{Signal yield correction}
\label{chapter3:sec:dimuons-sig-corr}

\section{Selection criteria}
\label{chapter3:sec:selection-criteria}

\subsection{Invisible mode}
\label{chapter3:sec:selection-criteria-invis}

\subsection{Visible mode}
\label{chapter3:sec:selection-criteria-vis}

\subsubsection{Analysis using Veto at the end of the dump}
\label{chapter3:sec:vis-mode-veto}

\subsubsection{Mixed approach using tracking and Veto}
\label{chapter3:sec:vis-mode-tracking}

%%% Local Variables:
%%% mode: latex
%%% TeX-master: t
%%% End:
