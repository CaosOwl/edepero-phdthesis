% Chapter 1

% cross references

% \chapter{Why Dark Matter?} % Main chapter title
\chapter{Introduction} % Main chapter title

\label{chapter6}

%----------------------------------------------------------------------------------------

\section{Evidence for Dark Matter}
\label{ch1:sec:dm-evidence}

Dark sectors are very interesting candidates to explain the origin of Dark Matter (see e.g. \cite{mb} for a recent review), whose presence has 
so far been inferred only through its gravitational interaction from cosmological observations \cite{hooper}. 
If, in addition to gravity, a new force between the dark sector and visible matter exists \cite{prw, pospelov} this can be tested in laboratory experiments. A possibility is that this new force is carried by a vector boson $A'$, called dark photon.
Stringent limits on the coupling strength $\epsilon$ and mass $m_{A'}$ of such dark photons, excluding the parameter space region favored by the anomalous magnetic moment of the muon, the so called $(g-2)_{\mu}$ anomaly, have already been placed by beam dump \cite{jdb, charm, rio, e137, konaka, bross, dav,  ath, nomad, e787, essig1, blum,sg1, blum1, sarah1}, fixed target \cite{apex,merkel,merkel1}, collider \cite{babar, curt, babar1}, rare particle decay searches \cite{sindrum, kloe, sg2, kloe2, wasa, hades, phenix, e949, na48, pol, kloe3} and the new determination of the fine structure constant $\alpha$ combined with the measurement of $(g-2)_e$ \cite{Parker191,PhysRevLett.100.120801}. 

\section{Dark Matter candidates}
\label{ch1:sec:dm-candidates}

\subsection{Thermal Dark Matter}
\label{ch1:sec:dm-thermal}

\subsection{Dark Matter in colliders}
\label{ch1:sec:dm-colliders}

\section{The U'(1) model}
\label{ch1:sec:dm-u1model}

\subsection{Motivations}
\label{ch1:sec:dm-u1model-motivations}

\subsubsection{The anomalous magnetic moment (g-2)$_{\mu}$}
\label{ch1:sec:dm-u1model-motivations-g2}

\subsubsection{The X17 anomaly}
\label{ch1:sec:dm-u1model-motivations-x17}

A great boost to search for the new light boson weakly coupled to Standard Model particles was triggered by the recent observation of a $\sim$7$\sigma$ excess of events in the angular distribution of $\pair$ pairs produced in the nuclear transitions of the excited $^8$Be$^*$ nuclei to its ground state via internal $\ee$ pair creation \cite{Krasznahorkay:2015iga}. The latest results of the ATOMKI group report a similar excess at approximately the same invariant mass in the nuclear transitions of another nucleus, $^4$He \cite{Krasznahorkay:2019lyl}.

It was put forward  \cite{Feng:2016jff,PhysRevD.95.035017}, that this anomaly can be interpreted as the emission of a protophobic gauge boson $\DM$ decaying into $\pair$ pairs. To be consistent with the existing constraints, the $\DM$ boson should have a non-universal coupling to quarks and a coupling strength with electrons in the range of $2\times 10^{-4} \lesssim \epsilon_e \lesssim 1.4\times 10^{-3}$ which translates to a lifetime of the order of $10^{-14}\lesssim \tau_X \lesssim 10^{-12}$~s. Remarkably, this model also explains within experimental uncertainty the new result obtained with the $^4$He nucleus, providing both kinematical and dynamical evidence to support this interpretation \cite{Feng:2020mbt}. This model will be used as benchmark for the NA64 current results and to cast the sensitivity of the new setup described in this article. However, other solutions of the $\DM$ anomaly were proposed, see for example \cite{Nam:2019osu, Seto:2016pks}.

Interestingly, such a new boson with a relatively large coupling to charged leptons could also resolve the tension between measured and predicted values of the $(g - 2)_{\mu}$. In addition to vector and axial-vector explanation of the $\DM$ anomaly, one can consider scenarios involving hidden pseudo-scalar boson \cite{Ellwanger:2016wfe}. Corresponding pseudo-scalar couplings to electrons satisfy existing experimental constraints \cite{Andreas:2010ms,Adler:2004hp}. An analysis to probe such pseudo-scalar states at NA64 \cite{Kirpichnikov:2020tcf} would require a proper Monte-Carlo simulation of the spectra and flux of light pseudo-scalar boson produced in the target by electrons.
Another interesting result comes from the new measurement of $\alpha$ performed by Parker et al. \cite{Parker191} which combined with the $(g-2)_e$ measurements result in a 2.4$\sigma$ deviation from the QED predictions \cite{PhysRevLett.100.120801}. Should this tension be confirmed, the two constraints coming from the NA64 results and $(g - 2)_e$ would exclude the vector and axial vector couplings explanation of $\DM$. On the other hand, models with nonzero V$\pm$A coupling constant with the electron would explain both electron and muon $(g - 2)$ anomalies \cite{Krasnikov:2019dgh}. In these models, the $\DM$ could have a coupling of $6.8\cdot 10^{-4} \lesssim \epsilon \lesssim 9.6 \cdot 10^{-4}$ which leaves an interesting region of the parameter space to be explored.
These models motivated the study of the phenomenological aspects of such a light vector boson weakly coupled to quarks and leptons (see, e.g., Refs.~\cite{fayet1, fayet2, fayet3, fayet4,jk, cheng, Zhang:2017zap, ia, liang, bart}) 
and new experimental searches (see e.g., Refs.~\cite{mb, nardi}).

Recently, the NA64 collaboration has reported new results that excluded the $\DM$ boson  with the coupling strength  to electrons in the range $1.2 \times 10^{-4} < \epsilon_e < 6.8 \times 10^{-4}$ \cite{Banerjee:2018vgk,Banerjee:2019hmi}, by using the calorimeter technique proposed in \cite{Gninenko:2013rka,Andreas:2013lya}. In this work, the main challenges to search for large coupling $\epsilon \sim 10^{-3}$ of $\DM$ will be outlined and an upgrade of the setup to overcome them is described. First, in Sec.\ref{ch3:sec:vis-mode-veto} an overview on the calorimeter method \cite{Gninenko:2013rka,Andreas:2013lya,Banerjee:2019hmi} is presented and the main limitations of the current setup are outlined. In Sec.\ref{ch3:sec:vis-mode-tracking}, a new analysis method that exploits the trackers is presented. This analysis highlights the importance of an efficient tracking procedure for the $\DM$ search. The increase in sensitivity is however negligible due to the intrinsic limitations of the setup.

\subsection{Decay modes}
\label{ch1:sec:dm-decay}

\subsubsection{Invisible decay mode}
\label{ch1:sec:dm-decay-invis}

\subsubsection{Visible decay mode}
\label{ch1:sec:dm-decay-vis}

%%% Local Variables:
%%% mode: latex
%%% TeX-master: "../PhDthesis"
%%% End:
