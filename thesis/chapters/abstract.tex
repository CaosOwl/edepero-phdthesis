The Standard Model of particle physics is a very successful framework used times and again by physicists to cast very accurate predictions. The recent discovery of the Higgs boson provided the last missing piece of this model, finally justifying a mechanism that generates the mass of elementary particles. However, there are still unsolved questions that require new physics. The origin of Dark Matter remains a mystery that has not been solved either by the Large Hadronic Collider or direct searches in the underground and astrophysical experiments. It was put forward the idea that Dark Matter is a part of a Dark Sector, it could be light, in sub-\si{\giga\electronvolt} mass range, and interact with the visible matter in addition to gravity through a new interaction transmitted by a dark mediator(s).  
Surprisingly and excitingly, the coupling strengths and masses of the latter, which could explain the observed relic abundance of Light Dark Matter, fall into the region  which can be probed in experiments at the current accelerators,  in particular for the vector case commonly called Dark Photon. The NA64 experiment aims to effectively cover this parameter space by using a \SI{100}{\giga\electronvolt} electron beam from the CERN SPS directed at an active fixed target. This approach is sensitive to different decay modes of the Dark photon with minimal modifications of the setup. In this thesis, we describe the two apparatus used and present the analysis of the data collected during 2016-2018. The focus will be on background rejection techniques and their application in the context of the NA64 experiment. The possibility of using the data collected to probe a new particle called $X17$, used to justify the recently observed deviation in the $^8$Be nuclear spectrum, is also detailed. NA64 already puts strong constraints on particle physics explanations of this anomaly, which are confirmed by an independent analysis performed in this thesis using tracking detectors. The remaining region of parameter space still able to justify the existence of a new boson in this context is hard to access due to the very short decay time this particle has for large couplings. In the last part of this thesis, we will discuss the upgrades that are under development to build the new generation of this experiment. The new setups will be able to cover the $\DMX$ parameter space completely and probe different models able to explain the observed relic density in the popular framework of thermal Dark Matter.
