The Standard Model of particle physics is a very successful framework used times and again by physicists to cast very accurate predictions. The recent discovery of the Higgs boson provided the last missing piece of this model, finally justifying a mechanism that generates the mass of elementary particles. However, there are still unsolved questions that require new physics. The origin of Dark Matter remains a mystery that has not been solved either by large collider or direct searches aimed to detect scattering with the Dark Matter halo in our galaxy.  It was put forward that Dark Matter can be explained by a Dark Sector interacting with the particles of the Standard model through a new portal interaction. The mediator of this interaction, commonly called Dark Photon in the case of a vector boson, can have a mass $< \SI{1}{\giga\electronvolt}$, which is challenging to probe using direct detection experiments. NA64 aims to cover this parameter space with a beam of \SI{100}{\giga\electronvolt} electrons directed at an active target. This approach is sensitive to different decay modes of the Dark photon with minimal modifications of the setup. In this thesis, we describe the two apparatus used and present the analysis of the data collected during 2016-2018. The focus will be on background rejection techniques and their application in the context of the NA64 experiment. The possibility of using the data collected to probe the so-called $\DMX$-anomaly is also detailed. NA64 already puts strong constraints on particle physics explanations of this anomaly, which are confirmed by an independent analysis performed in this thesis using tracking detectors. The remaining region of parameter space still able to justify the existence of a new boson in this context is hard to access due to the very short decay time this particle has for large couplings. In the last part of this thesis, we will discuss the upgrades that are under development to build the new generation of this experiment. The new setups will be able to cover the $\DMX$ parameter space completely and probe different models able to explain the observed relic density in the popular framework of thermal Dark Matter.
