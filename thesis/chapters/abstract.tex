The Standard Model of particle physics has provided incredibly accurate predictions confirmed by experimental data. The recent discovery of the Higgs boson at the Large Hadron Collider (LHC) in 2012 was another remarkable landmark of this model. However, despite its success the Standard Model cannot address some open questions and therefore new physics is required. Among those is the origin of Dark Matter which remains a mystery that has not been solved either by LHC or direct searches in the underground and astrophysical experiments. It was put forward that if Dark Matter is part of a Dark Sector, it could be light, in sub-\si{\giga\electronvolt} mass range, and interact with the visible matter in addition to gravity through a new interaction transmitted by a dark mediator(s).  
Surprisingly and excitingly, the coupling strengths and masses of the latter, which could explain the observed Dark Matter relic abundance, fall into the region  which can be probed in experiments at current accelerators,  in particular for the vector case commonly called Dark Photon. The NA64 experiment aims to effectively cover this parameter space by using a \SI{100}{\giga\electronvolt} electron beam from the CERN Super Proton Synchrotron (SPS) directed at an active fixed target. This approach is sensitive to different decay modes of the Dark photon with minimal modifications of the setup. In this thesis, we describe the two apparatus used and present the analysis of the data collected during 2016-2018. The focus will be on background rejection techniques and their application in the context of the NA64 experiment. The possibility of using the data collected to probe the existence of a new particle called $X17$, used to justify the recently observed deviation in the $^8$Be nuclear spectrum, is also detailed. NA64 already puts strong constraints on particle physics explanations of this anomaly, which are confirmed by an independent analysis performed in this thesis using tracking detectors. The remaining region of parameter space still able to justify the existence of a new boson in this context is hard to access due to the very short decay time this particle has for large couplings. This work will also present a new setup optimised for these searches. The future prospects of the experiment and the foreseen upgrades will be discussed in the last section. The new setup will allow to fully cover the $\DMX$ parameter space.
