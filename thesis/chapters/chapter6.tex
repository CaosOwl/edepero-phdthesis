% Chapter 6

\chapter{Conclusion}

\label{chapter6}

Despite its great success, the Standard Model is not complete. One of the most compelling open questions that cannot be addressed by the SM is the origin of DM. Many possible candidates have been put forward. In this thesis, the focus is set on an interesting class of models which are called Dark Sectors. Those were introduced in chapter $\ref{chapter1}$ and the emphasis was put on the model predicting an additional vector gauge boson called Dark photon which can be probed in the NA64 experiment at the CERN SPS. In chapter \ref{chapter2} the NA64 experiment was described in detail. It was shown that the setup can be adapted to search for two different decay channels of the Dark Photon: the invisible decay mode detected by searching for missing energy in the final state and the visible mode where an $\ee$ pair should be detected after a thick target. The author was responsible for the spectrometer composed of a dipole magnet and a tracker. This was used to ensure that the incoming particle momentum was compatible with the nominal beam energy. The other main contribution was the development of the Synchrotron radiation detector, which together with a shower profile analysis of the ECAL allowed to identify the incoming electrons discriminating those from hadron and muon contamination in the beam.

In chapter $\ref{chapter3}$ we outlined the analysis performed for the data accumulated in the period 2016-2018. Both Monte Carlo techniques and data-driven methods were used to characterize the background of the experiment, demonstrating that this was under control for the accumulated statistics of $\sim 3 \times 10^{11}$ EOT for the invisible mode and $\sim 9 \times 10^{10}$ EOT for the visible one. The background expected was 0.6 events, making the experiment background free for the statistics collected during this work. The author contributed heavily to the development of the software necessary to analyze the data, to produce a detailed MC simulation of the experiment including the Dark Photon production. The focus of this thesis was set on the visible mode and in particular the exploration of the parameter space justifying the $\DMX$. This new boson could provide an explanation of the observed anomaly in the nuclear spectrum of both $^8$Be and $^4$He atoms. The author participated in the background estimation and helped to define the selection criteria to maximize the sensitivity of the experiment. Additionally, the author performed an independent analysis using the information of the trackers in the decay volume. The improvement of this analysis was found to be marginal but confirmed the previous results based on a pure calorimetry analysis. Furthermore, it served as an important proof of principle for a tracking approach that will be essential in the next generation of this setup.

No event was detected inside the signal region in neither of the data collected with the two different setups. This allowed the exclusion of several models that were in the sensitivity range of the NA64 experiment. In chapter $\ref{chapter4}$, we gave a summary of the achieved results by showing the range of models excluded to date. For the invisible decay of $\DM$, NA64 covers a significant portion of the parameter space relevant for the Dark photon, including the band that would justify the anomalous magnetic moment $\ammu$ of the muon.
The data collected in the invisible mode was also used to cover an unexplored region of parameter space for Axion-like and light scalar particles. Many more models could also be constrained and will be the focus of future analyses. Examples are pseudo-scalar gauge boson or the semi-visible decay $\DM \to \chi_1 \chi_2 (\chi_2 \to \ee)$ \cite{Mohlabeng_2019}. In this last scenario, the band explaining the $\ammu$ would not be yet completely excluded, which makes this parameter space especially interesting. A pseudo-scalar mediator, on the other hand, could cover a parameter space relevant for the $\DMX$-anomaly characterized by small coupling $\epsilon_e \lesssim 10^{-4}$, cross-checking the results from the E141 experiment \cite{blum}.

The analysis of the visible mode data was also used to cover a significant portion of unexplored parameter space. In particular, NA64 is sensitive to models capable of explaining the $\DMX$. Considering the protophobic vector boson proposed by \textit{Feng et al.}\cite{Feng:2016jff} as benchmark model, NA64 excluded couplings in the range $2 \times 10^{-5} \lesssim \epsilon_e \lesssim 6.8 \times 10^{-4}$. This result leaves a range $6.8 \times 10^{-4} \lesssim \epsilon_e \lesssim 1.4 \times 10^{-3}$ left to explore.

In chapter $\ref{chapter5}$ we presented the future prospects for the NA64 experiment. Accumulating a larger number of EOT, estimated as $5 \times 10^{12}$, would extend the sensitivity of the experiment to the region compatible with the observed relic abundance in a freeze-out scenario. This possibility was studied and the conclusion is that background from events in which electro-hadron production upstream of the target with large scattering angles will become relevant. To suppress this source, a reduction of the material of the Micromegas detector is needed. The author developed in this thesis a new design to achieve this purpose, where the Copper seal of the gas box is substitute with an aluminized Kapton foil. Additionally, the multiplexing map of this detector was also improved to reduce the redundancies from double hits, a feature that will be necessary to track the decay $\aee$ in the visible mode setup. Finally, a new calorimeter (VHCAL) will be included to veto electro-nuclear interaction along the beamline. Considering all these contributions, the author performed a study for the level of this background including also Straw chambers, implemented for the first time in the simulation. The estimated background for 5$\times 10^{12}$ EOT is of 5$\pm$1 events in the signal box.

The author was also deeply involved in the design of a new setup for the visible mode. The small decay length $<$30 $\mmi$ and the low angle of the decay products $\angee < 0.3$ $\mrad$ suppress the signal yield exponentially. A new design was studied to specifically cover short-lived particles. The total length of the dump was reduced without changing the overall radiation length, to keep the background under control. At the same time, a long vacuum tube is placed after the end of that target, with an additional magnetic spectrometer placed at the end of the decay volume. This setup ensures a resolution of both angle and momentum of $<1\%$ for the $\ee$ emitted in the $\DMX$ decay, allowing the reconstruction of the invariant mass with a precision $\sim$2\%. Using a detailed MC-simulation of the setup, a total number of $8 \times 10^{11}$ was estimated to be necessary to cover the largest allowed coupling $\epsilon_e \sim 1.4 \times 10^{-3}$.

In 2021, NA64 will resume data taking with many challenges to overcome but also an unprecedented sensitivity to new physics. These thesis results were crucial for the design of the new setups and to have a realistic estimate of the projected sensitivities and background sources. Exciting times are ahead for Dark Sectors searches.
%----------------------------------------------------------------------------------------


%%% Local Variables:
%%% mode: latex
%%% TeX-master: "../PhDthesis"
%%% End: