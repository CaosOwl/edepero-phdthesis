% Chapter 6

\chapter{Conclusion} % Main chapter title

\label{chapter6}

With this work, I hope I manage to provide a complete and detailed picture of the work done by the NA64 experiment to search for Dark Matter using the SPS structure provided at CERN.

We started this thesis by exploring the many possibilities that are left to explain the Dark matter puzzles, and we went into details regarding the dark sector and specifically the model predicting an additional vector gauge boson called Dark photon. In chapter \ref{chapter2} we described the experiment designed by the NA64 collaboration to hunt down this particle and possibly other ones with the same properties. We saw that both setup and background condition change if we try to probe different decay channels of this particles, namely if we assume the main branching ratio decays invisibly in the dark sector or visibly in $\ee$ pair. In both cases, one has to ensure the incoming particle to be an electron with momentum compatible with the nominal beam energy. Synchrotron radiation was used to discriminate the incoming particle mass together with shower profile analysis used to distinguish between electromagnetic and hadronic shower. The momentum on the other hand, was measured using a mass spectrometer made using 6-8 Micromegas tracking chamber and two MBPL magnets for a total of $\sim$7 \si{\tesla\meter}.

The analysis was performed using both Monte Carlo techniques and data driven method to characterize the background of the experiment, that was kept under control for the statistic accumulated to date of $\sim 3 \times 10^{11}$ for the invisible mode and $\sim 9 \times 10^{10}$ for the visible one.

To date, no event was detected inside the signal region in neither of the experiments. To our best knowledge, model predicting a dark sector cannot be used to explain the Dark Matter puzzle yet. Nevertheless, the data collected allowed the exclusion of several models of dark matter that were in the sensitivity's  range  of the NA64 experiment. 
%----------------------------------------------------------------------------------------


%%% Local Variables:
%%% mode: latex
%%% TeX-master: "../PhDthesis"
%%% End:
