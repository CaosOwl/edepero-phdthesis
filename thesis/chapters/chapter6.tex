% Chapter 6

\chapter{Conclusion} % Main chapter title

\label{chapter6}

With this work, my hope is that I managed to provide the reader a complete and detailed description of the work done by the NA64 collaboration and me in particular to search for light Dark Matter using the SPS at CERN.

We started this thesis by describing the many possibilities left to explain the Dark matter puzzles, and we went into details regarding the Dark Sector and specifically the model predicting an additional vector gauge boson called Dark photon. In chapter \ref{chapter2} the NA64 experiment was described in detail, and we saw that the setup can be adapted to the different decay channels of this particle. Two different setups are used depending on the main branching ratio assumed: an invisible decay is detected by searching missing momentum in the final state, while a $\ee$ pair after a thick target is required for a visible decay. In both cases, one has to ensure the incoming particle to be an electron with momentum compatible with the nominal beam energy. Synchrotron radiation is used together with a shower profile analysis to discriminate the incoming particle ID. The momentum is then measured using a mass spectrometer made of 6-8 Micromegas tracking chamber and two MBPL magnets.

The analysis was performed using both Monte Carlo techniques and data-driven methods to characterize the background of the experiment, that was kept under control for the statistic accumulated to date of $\sim 3 \times 10^{11}$ EOT for the invisible mode and $\sim 9 \times 10^{10}$ EOT for the visible. The background expected was never above 0.6 events, making the experiment background free for the statistics collected during this work.

No event was detected inside the signal region in neither of the setup for the analyses performed to date.The data collected allowed the exclusion of several models that were in the sensitivity range of the NA64 experiment. For the invisible decay of $\DM$, NA64 covers a significant portion of the parameter space relevant for the Dark photon, including the band that would justify the anomalous magnetic moment $\ammu$ of the muon. Accumulating a larger number of EOT, estimated as $5 \times 10^{12}$, would extend the sensitivity of the experiment to the region compatible to the observed relic abundance in a freeze-out scenario. We studied this possibility and concluded that background from large scattering electro-hadron production upstream the target becomes relevant. Some strategies to limit this source of background were proposed, including a reduction of the material of the Micromegas detector, and the introduction of a new calorimeter to veto such interactions. Including all these contributions, some uncertainty for this background is still expected, with estimates ranging from 5$\pm$1 to 0.5$\pm$0.2 events in the signal box.

The data collected for the invisible mode was used to cover an unexplored region of parameter space for Axion-like and light scalar particles as well. The sensitivity for these models will also increase from the higher statistic planned for future runs. Many more models can also be constrained with the data collected, examples are pseudo-scalar gauge boson or the semi-visible decay $\DM \to \chi_1 \chi_2 (\chi_2 \to \ee)$ \cite{Mohlabeng_2019}. In this last scenario, the band explaining the $\ammu$ would not be yet completely excluded, which makes this parameter space especially interesting. A pseudo-scalar mediator, on the other hand, could cover a parameter space relevant for the $\DMX$-anomaly characterized by small coupling $\epsilon_e \lesssim 10^{-4}$, cross-checking in the process the results from the E141 experiment \cite{blum}.

The data collected for the visible mode were also used to cover a significant portion of unexplored parameter space. In particular, NA64 is sensitive to models capable of explaining the $\DMX$-anomaly detected in the ATOMKI institute. Considering the protophobic vector boson proposed by \textit{Feng et al.}\cite{Feng:2016jff} as benchmark model, NA64 excluded couplings in the range $2 \times 10^{-5} \lesssim \epsilon_e \lesssim 6.8 \times 10^{-4}$. This results leaves a range $6.8 \times 10^{-4} \lesssim \epsilon_e \lesssim 1.4 \times 10^{-3}$ left to explore. As shown in this work, using a novel analysis that includes the trackers in the decay volume, covering the remaining region of parameter space is unfeasible in the current setup. The small decay length $<$30 $\mmi$ and the low angle of the decay $\angee < 0.3$ $\mrad$ suppress the signal yield exponentially, reducing the sensitivity of the experiment dramatically. A new design was presented to specifically cover short-lived particles. The total length of the dump was reduced without changing the overall radiation length, to keep the background under control. At the same time, a long vacuum tube is placed after the end of that target, with an additional magnetic spectrometer placed at the end of the decay volume. This setup ensures a resolution of both angle and momentum of $<1\%$ for the $\ee$ emitted in the $\DMX$ decay, allowing the reconstruction of the invariant mass with a precision $\sim$2\%. Using a detailed MC-simulation of the setup, a total number of $8 \times 10^{11}$ was estimated to be necessary to cover the largest allowed coupling $\epsilon_e \sim 1.4 \times 10^{-3}$.

2021 will be an exciting year for the NA64 experiment, with many challenges ahead but also an unprecedented sensitivity to new physics. I hope that this work showed the power of the techniques presented to probe many different Dark Matter candidates. Particle physics at accelerators once again proved to be an invaluable tool to probe new physics, and it will likely remain so for the years to come.
%----------------------------------------------------------------------------------------


%%% Local Variables:
%%% mode: latex
%%% TeX-master: "../PhDthesis"
%%% End:
