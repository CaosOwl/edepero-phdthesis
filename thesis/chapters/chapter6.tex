% Chapter 6

\chapter{Conclusion} % Main chapter title

\label{chapter6}

With this work, I hope I manage to provide a complete and detailed picture of the work done by the NA64 experiment to search for Dark Matter using the SPS structure provided at CERN.

We started this thesis by exploring the many possibilities that are left to explain the Dark matter puzzles, and we went into details regarding the dark sector and specifically the model predicting an additional vector gauge boson called Dark photon. In chapter \ref{chapter2} we described the experiment designed by the NA64 collaboration to hunt down this particle and possibly other ones with the same properties. We saw that both setup and background condition change if we try to probe different decay channels of this particle, namely if we assume the main branching ratio decays invisibly in the dark sector or visibly in $\ee$ pair. In both cases, one has to ensure the incoming particle to be an electron with momentum compatible with the nominal beam energy. Synchrotron radiation was used to discriminate the incoming particle mass together with shower profile analysis used to distinguish between electromagnetic and hadronic shower. The momentum, on the other hand, was measured using a mass spectrometer made using a 6-8 Micromegas tracking chamber and two MBPL magnets for a total of $\sim$7 \si{\tesla\meter}.

The analysis was performed using both Monte Carlo techniques and data-driven methods to characterize the background of the experiment, that was kept under control for the statistic accumulated to date of $\sim 3 \times 10^{11}$ EOT for the invisible mode and $\sim 9 \times 10^{10}$ EOT for the visible one. The background expectation never exceeded 0.6 events in the signal box, making the experiment background free for the statistics collected to this date.

No event was detected inside the signal region in neither of the experiments for the analysis performed up to the time this thesis was written. To our best knowledge, models predicting a dark sector cannot be used to explain the Dark Matter puzzle yet. Nevertheless, the data collected allowed the exclusion of several models of dark matter that were in the sensitivity range of the NA64 experiment. To cast out results, different models of dark sector were considered. A new vector gauge boson generated by a new $U'(1)$ symmetry was originally taken as a benchmark model to interpret the experiment sensitivity. In the case where the $\DM$ decays invisibly, the data collected by NA64 already cover a significant portion of the relevant parameter space, including the band that would justify the anomalous magnetic moment of the muon. Accumulating a larger number of EOT, estimated as $5 \times 10^{12}$, would extend the sensitivity of the experiment to the band justifying the observed relic abundance in a freeze-out scenario. We studied this possibility in this work and concluded that background from large scattering electro-hadron production upstream the target becomes relevant. Some strategies to limit this source of background were proposed, including a reduction of the material of the Micromegas detector, and the introduction of a new calorimeter to veto such interactions. Including all these contributions, some uncertainty for this background is still expected, with estimates ranging from 5$\pm$1 to 0.5$\pm$0.2 events in the signal box.

The data collected for the invisible mode were used to cast sensitivity for additional models interesting to explain the Dark Matter puzzle, like for example axion-like and scalar particles. Sensitivity for these models will also increase from the higher statistic planned for future runs. Many more models can also be constrained with the data collected, examples are pseudo-scalar gauge boson or the semi-visible decay $\DM \to \chi_1 \chi_2 (\chi_2 \to \ee)$ predicted by models with large mass splitting \cite{Mohlabeng_2019}. In this last scenario, the band explaining the $\ammu$ would not be yet completely excluded, which makes this parameter space particularly interesting. A pseudo-scalar mediator, on the other hand, could cover a parameter space relevant for the $\DMX$-anomaly characterized by small coupling $\epsilon_e \lesssim 10^{-4}$, cross-checking in the process the results from the E141 experiment \cite{blum}.

The data collected for the visible mode were also used to cover a significant portion of unexplored parameter space. In particular, NA64 is one of the few experiments sensitive to the particle physics model capable to explain the $\DMX$-anomaly detected in the ATOMKI institute. Considering the protophobic vector boson proposed by \textit{Feng et al.}\cite{Feng:2016jff} as benchmark model, NA64 excluded couplings in the range $2 \times 10^{-5} \lesssim \epsilon_e \lesssim 6.8 \times 10^{-4}$, this results leaves a range $6.8 \times 10^{-4} \lesssim \epsilon_e \lesssim 1.4 \times 10^{-3}$ left still allowed. As shown in this work, using a novel analysis that includes the trackers in the decay volume, the current setup is fundamentally incapable of covering the remaining region of parameter space. The small decay length $<$30 \mmi and the low angle of the decay $\angee < 0.3$ \mrad suppress the signal yield exponentially, dramatically reducing the sensitivity of the experiment. A new design was presented to specifically target short-lived particles. The total length of the dump was reduced without changing the overall radiation length, to keep the background under control. At the same time, a long vacuum tube is placed after the end of that target, with an additional magnetic spectrometer placed at the end of the decay volume. This setup ensures a resolution of both angle and momentum of $<1\%$ for the $\ee$ emitted in the $\DMX$ decay, allowing the reconstruction of the invariant mass at a level of $\sim$2\%. The invariant mass can be then matched to the mass observed by \textit{Krasznahorkay et al.}, to demonstrate unequivocally a particle physics explanation of the anomaly. Using a detailed MC-simulation of the setup, a total number of $8 \times 10^{11}$ was estimated to be necessary to cover the largest allowed coupling $\epsilon_e \sim 1.4 \times 10^{-3}$.

2021 will be an exciting year for the NA64 experiment, with many challenges ahead but also an unprecedented sensitivity to new physics. I hope that this work was sufficient to show the usefulness of the techniques presented to probe many different candidates that would offer a solution to the Dark Matter puzzle. Physics at collider once again proved to be an invaluable tool to probe new physics, and it will likely remain so for the years to come.
%----------------------------------------------------------------------------------------


%%% Local Variables:
%%% mode: latex
%%% TeX-master: "../PhDthesis"
%%% End:
