Il Modello Standard è stato utilizzato nel contesto della fisica delle particelle per fornire predizioni incredibilmente accurate confermate poi dai dati raccolti sperimentalmente. La recente scoperta del bosone di Higgs presso il Large Hadron Collider (LHC) nel 2012 è stato un ulteriore incredibile successo di questa teoria. Nonostante ciò, il Modello Standard non è in grado di spiegare alcune domande aperte e pertanto rimane il bisogno di estendere questa teoria con della nuova fisica. Tra queste, l'origine della materia oscura rimane un mistero irrisolto nonostante le numerose ricerche effetuate sia con il LHC che dagli esperimenti ad osservazione diretta installati sottoterra o dalle numerose osservazioni in ambito astrofisico. Recentemente è stato proposto che la materia oscura sia parte di un Settore Oscuro, potenzialmente popolato da particelle leggere, di masse inferiori al \si{\giga\electronvolt}, che potrebbe interagire con la materia visibile non solo tramite gravità ma anche tramite una nuova interazione mediata da un bosone oscuro.
Sorprendentemente, la forza di interazione e la massa di questo bosone, in grado di spiegare l'abbondanza di Materia Oscura osservata, si trovano in una regione dello spazio dei parametri che può essere esplorata tramite i nostri attuali accelleratori, in particolare nel caso di un bosone vettoriale chiamato Fotone Oscuro.
L'esperimento NA64 ha come obiettivo il coprire questo spazio di parametri dirigendo un fascio di elettroni di energia \SI{100}{\giga\electronvolt} generato dal Super Proton Synchrotron (SPS) al CERN verso un target attivo. Questo metodo è sensibile a differenti tipi di decadimento del fotone oscuro con modifiche minime all'apparato sperimentale. In questa tesi, presentiamo l'apparato sperimentale usato e l'analisi dei dati raccolti durante il periodo 2016-2018. Il focus sarà sulle tecniche utilizzate per ridurre il fondo e la loro applicazione nel contesto di NA64. Viene qui anche discussa la possibilità di usare questi dati per verificare l'esistenza di una nuova particella chiamata $X17$, proposta per giustificare una anomalia recentemente osservata nello spettro nucleare del atomo $^8$Be. NA64 è già in grado di limitare significamente le possibili spiegazioni di questa anomalia nella fisica delle particelle, in questa tesi una ulteriore conferma viene data da un analisi indipendente effettuata usando i rilevatori di traccia. La regione dello spazio dei parametri ancora non coperta è difficile da accedere per via del tempo di decadimento molto corto di questa particella quando la sua forza di interazione cresce. In questa tesi viene anche presentato un nuovo apparato sperimentale ottimizato per queste ricerche. Nell'ultima sezione verranno discusse le prospettive future di questo esperimento e diverse migliorie previste per il futuro. Il nuovo apparato permetterà di coprire completamente lo spazio di parametri in grado di giustificare la particella $X17$.